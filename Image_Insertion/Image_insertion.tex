\documentclass{article}

\usepackage{graphicx}
\usepackage{subcaption}
\usepackage{float}
 
% the above were package of the niserting graphics

\begin{document}
	\begin{figure}
		\includegraphics[width=\linewidth]{Desktop.jpeg}
		\caption{Desktop}
		\label{fig:Desktop}
	\end{figure}

	Figure \ref{fig:Desktop} An inspiration quote
	% shifting to another photo
	\begin{figure}[h!]
		\centering
		\begin{subfigure}[t!]{0.4\linewidth}
		\includegraphics[width=\linewidth]{robo_arm.jpg}
		\caption{Robotic Arm of type - 1}
	\end{subfigure}
	\begin{subfigure}[t!][0.4\linewidth]
		\includegraphics[width=\linewidth]{robo_arm_1.jpg}
		\caption{Robotic Arm of type - 2}
	\end{subfigure}
	\caption{Different types of Robotic arm}
	\label{fig:Robotic Arm}
	\end{figure}
	\begin{figure}[H]
		\centering
		\begin{subfigure}[t]{0.4\linewidth}
		\includegraphics[width=\linewidth]{robo_arm.jpg}
		\caption{Robotism}
	\end{subfigure}
	\begin{subfigure}[t]{0.4\linweidth}
		\includegraphics[width=\linewidth]{robo_arm_1.jpg}
		\caption{Robotism-2}
	\end{subfigure}
	\begin{subfigure}[t]{0.5\linewidth]
		\includegraphics[width=\linewidth]{Personal.png}
		\caption{1.png}
	\end{subfigure}
	\caption{Some picture experiment}
	\label{fig:more}
	\end{figure}
\end{document}